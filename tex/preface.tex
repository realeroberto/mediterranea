Mi vengono giorni che scrivere di Mediterraneo mi fa fatica, che sguardo e penna hanno voglia di slarghi in cielo, di panelachi, di guglie e di ponti. Invece quaggiù stringe troppo la denuncia di sé sempre esatta, estorta nella controra, la persistenza del dio.

Ma non importa, l'esercizio è da tentare comunque. Smemoratezza, distacco, fuga. Ma poi. È roba nostra comunque questo mare, e sarebbe scappare da sé: allora migliore è portarselo dentro sempre e goderne o soffrirne quando è il suo tempo, e scriverne.
